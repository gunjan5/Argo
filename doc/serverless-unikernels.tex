%$Id:$
\documentclass[10pt]{article}
\usepackage{times}
\usepackage{epsfig,verbatim}
\usepackage{cite,mathptm}

\begin{document}

\title{Serverless architecture with Unikernels}

\author{ 
Gunjan Patel\\
{\it gunjanpatel.v@gmail.com}
}

\maketitle

\begin{abstract}
{\it
\input{abstract}

Not {\bf in} the abstract file!!!!!

}
\end{abstract}

\section{Introduction}
\label{sec:intro}

This is \S\ref{sec:intro}.


\noindent In-line math mode: $y = x^{\sqrt{2^{x+y}}}_{\phi \Phi}$.

An equation array:
\begin{eqnarray}
x & = & \sqrt{y} \\
  & = & z^{y+d_{(i+j)}} \nonumber \\
  & = & 12
\end{eqnarray}

If using {\em this} template, please rename ``student'' in the file names
to your login name in the cs.pitt.edu domain. That'll make it easier to manage
multiple electronic submissions.
To build this file do the following:
\begin{verbatim}
$ make pdf
\end{verbatim}


\section{Related Work}
\label{sec:related}


Remember to cite papers on related work, but also remember
to cite papers that describe traces you use ({\it e.g.} 
Drew Roselli's technical report~\cite{roselli98}, when using 
the Berkeley traces). For your first use of this template ... just
a single section is sufficient, don't worry about paper structure yet.



\section{Conclusion}
\label{sec:conclusion}

%A comment in latex is preceded with a percentage sign

%The following two lines specify the bibliography file(s) used
% ... in this case student-doc.bib, and the style of the bibliography
% ... in this case ieee.bst

\bibliography{student-doc}
\bibliographystyle{ieee}


\end{document}
